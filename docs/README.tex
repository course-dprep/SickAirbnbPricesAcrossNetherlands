% Options for packages loaded elsewhere
\PassOptionsToPackage{unicode}{hyperref}
\PassOptionsToPackage{hyphens}{url}
%
\documentclass[
]{article}
\usepackage{amsmath,amssymb}
\usepackage{lmodern}
\usepackage{ifxetex,ifluatex}
\ifnum 0\ifxetex 1\fi\ifluatex 1\fi=0 % if pdftex
  \usepackage[T1]{fontenc}
  \usepackage[utf8]{inputenc}
  \usepackage{textcomp} % provide euro and other symbols
\else % if luatex or xetex
  \usepackage{unicode-math}
  \defaultfontfeatures{Scale=MatchLowercase}
  \defaultfontfeatures[\rmfamily]{Ligatures=TeX,Scale=1}
\fi
% Use upquote if available, for straight quotes in verbatim environments
\IfFileExists{upquote.sty}{\usepackage{upquote}}{}
\IfFileExists{microtype.sty}{% use microtype if available
  \usepackage[]{microtype}
  \UseMicrotypeSet[protrusion]{basicmath} % disable protrusion for tt fonts
}{}
\makeatletter
\@ifundefined{KOMAClassName}{% if non-KOMA class
  \IfFileExists{parskip.sty}{%
    \usepackage{parskip}
  }{% else
    \setlength{\parindent}{0pt}
    \setlength{\parskip}{6pt plus 2pt minus 1pt}}
}{% if KOMA class
  \KOMAoptions{parskip=half}}
\makeatother
\usepackage{xcolor}
\IfFileExists{xurl.sty}{\usepackage{xurl}}{} % add URL line breaks if available
\IfFileExists{bookmark.sty}{\usepackage{bookmark}}{\usepackage{hyperref}}
\hypersetup{
  pdftitle={README},
  hidelinks,
  pdfcreator={LaTeX via pandoc}}
\urlstyle{same} % disable monospaced font for URLs
\usepackage[margin=1in]{geometry}
\usepackage{graphicx}
\makeatletter
\def\maxwidth{\ifdim\Gin@nat@width>\linewidth\linewidth\else\Gin@nat@width\fi}
\def\maxheight{\ifdim\Gin@nat@height>\textheight\textheight\else\Gin@nat@height\fi}
\makeatother
% Scale images if necessary, so that they will not overflow the page
% margins by default, and it is still possible to overwrite the defaults
% using explicit options in \includegraphics[width, height, ...]{}
\setkeys{Gin}{width=\maxwidth,height=\maxheight,keepaspectratio}
% Set default figure placement to htbp
\makeatletter
\def\fps@figure{htbp}
\makeatother
\setlength{\emergencystretch}{3em} % prevent overfull lines
\providecommand{\tightlist}{%
  \setlength{\itemsep}{0pt}\setlength{\parskip}{0pt}}
\setcounter{secnumdepth}{-\maxdimen} % remove section numbering
\ifluatex
  \usepackage{selnolig}  % disable illegal ligatures
\fi

\title{README}
\author{}
\date{\vspace{-2.5em}}

\begin{document}
\maketitle

\hypertarget{how-covid-19-infected-the-airbnb-prices-in-amsterdam-the-netherlands}{%
\section{``How Covid-19 infected the Airbnb prices in Amsterdam, The
Netherlands''}\label{how-covid-19-infected-the-airbnb-prices-in-amsterdam-the-netherlands}}

\hypertarget{a-study-on-how-the-second-lockdown-in-the-netherlands-affected-the-airbnb-market-in-amsterdam}{%
\subsubsection{A study on how the second lockdown in the Netherlands
affected the AirBnb market in
Amsterdam}\label{a-study-on-how-the-second-lockdown-in-the-netherlands-affected-the-airbnb-market-in-amsterdam}}

Research question: To what extent does the second lockdown in the
Netherlands has changed the price per night of the Air bnb
accommodations in Amsterdam, and to what extent does the location of the
accommodation influence this effect?

\hypertarget{motivation}{%
\subsection{Motivation}\label{motivation}}

At the end of 2019, the first infection of COVID19 was detected in
China, ever since, the virus has gripped almost all countries in the
world
\href{https://www.frontiersin.org/articles/10.3389/fcell.2020.00476/full}{Chen
et al., 2020}.The Netherlands has also been affected by COVID19 and the
government has quickly taken measures to contain the damage and limit
the spread of the virus. In December 2020, the Netherlands entered the
second lockdown, which ended around May 2021. During this lockdown, the
Netherlands had to deal with far-reaching measures. Some examples of
these gripping measures were:

\begin{itemize}
\tightlist
\item
  Keep 1.5 meters distance from people who are not part of your
  household;
\item
  All non- essential visiting locations were closed;
\item
  Only use public transport if necessary;
\item
  For a limited period there was a curfew.
\end{itemize}

The above examples are just a few measures that were applied during the
second lockdown and it is almost inconceivable that these measures have
not had an impact on the AirBnb market. Therefore, this study will pay
attention to one of concequence of this second lockdown on the Airbnb
market in Amsterdam, namely price of the Airbnb accomondations.

Ultimately, this study will therefore answer the following questions: *
How did the second lockdown in the Netherlands affect the prices of Air
bnb accommodations in Amsterdam? * How did location (different districts
within Amsterdam) affect prices during this period (December 2020 - May
2021)? * In what district (within Amsterdam) did the price changed the
most?

This information provides insights of the role of COVID19 on the Air bnb
market. This gives landlords insight into the impact of such a pandemic
on the rental market. They can use this to anticipate more quickly to
the event of a possible next pandemic or to make choices about what to
do with the accommodation in such an uncertain times. COVID19 is not the
first and will not be the last pandemic that the Netherlands, or the
world, will experience (WHO, 2020). The information obtained from this
research will therefore help Air bnb landlords make choices in the next
pandemic.

\begin{figure}
\centering
\includegraphics{https://user-images.githubusercontent.com/89807582/136343144-d28c112a-9c36-4c83-8997-f9d549e1127f.jpg}
\caption{image1}
\end{figure}

\hypertarget{research-method}{%
\subsection{Research method}\label{research-method}}

\hypertarget{data}{%
\subsection{Data}\label{data}}

For this research project Airbnb data will be used, accessed through
\href{http://insideairbnb.com/get-the-data.html}{Inside Airbnb}.The
datasets used in this project are all based in the listings in
Amsterdam, The Netherlands and framed through the timeline 12 December
2020 including and to January 2021, February 2021, March 2021, April
2021 and 19 May 2021. These date frames are selected with accordance to
the second COVID-19 lock-down dates (14 December 2020 -11 May 2021) in
the Netherlands, where measures and restrictions are taken heavily
towards COVID-19 where after May 11 the relaxation steps of these
measures were taken.

\hypertarget{method-and-results}{%
\subsection{Method and results}\label{method-and-results}}

In this chapter we will give a brief overview of the methods that have
been applied, followed by the results from the ANOVA test and ends with
the conclusion to our research question.

\hypertarget{method}{%
\paragraph{Method}\label{method}}

First we started out by downloading and cleaning the data. We got the
data from Airbnb Inside and chose to focus on the impact of the second
lockdown in the Netherlands on the price per night of an Air bnb
accommodation in Amsterdam and the influence of the location on this. We
chose to pick the months December 2020 till May 2021 so we would have a
clear view of the months before the second lockdown (December and
January), during the lockdown (February and March) and after the second
lockdown (April and May).

We filtered out the columns that we needed, all related to price and
district and categorized them in seven districts: Oost-, Center-, Zuid-,
West-, Noord-, Nieuw West- and Zuidoost Amsterdam. The data of all
months were merged together. The ANOVA analysis fits our research
question best, with the independent variable being categorical (the
second lockdown in the Netherlands) and our dependent variable being
continuous (price per night). The ANOVA analysis estimates how the price
changed due to the second lockdown and what the influence of the
district is on this change. ANOVA tests whether there is a difference in
means of the groups at each level of the independent variable.

\hypertarget{results}{%
\paragraph{Results}\label{results}}

To answer the research question, the method that will be applied in this
research is one way ANOVA. One-way ANOVA (``analysis of variance'') is a
method used to compare the means of two or more groups in order to
determine whether or not the difference in the means of these groups are
statistically different. The price difference after the second lockdown
in Amsterdam, The Netherlands will be calculated manually but the main
relation of whether or not this price difference is significantly
different across the districts within Amsterdam is tested with one-way
ANOVA.

After applying the analysis to our merged data sets over the moths of
the second lockdown we have reached the following results:

The p- value (7.46e-10) is less then the significance level of 0.05.
Therefore we can conclude that there is a significance difference
between the different districts (variable: neighbourhood\_cleansed).

The ANOVA test assumes that, the data is normally distributed and the
variance between groups is homogeneous. We can check that with some
diagnostic plots. In the plot, there are no obvious relationships
between residuals and adjusted values (the mean of each groups), which
is good. So we can assume the homogeneity of variances. In order to come
to this conclusion, we decided to filter out the outliers that appeared
above the 6000.

In addition we also used the Levene's test to check the homogeneity of
the variances. The Pr(\textless F) output we got is 2.2e-16. From this
output we can see that the p-value is not less than the significance
level of 0.05. This means that there is no evidence to suggest that the
variance across groups is statistically significantly different.
Therefore, we can assume the homogeneity of variances in the different
treatment groups. Also, as all the points fall approximately along this
reference line, we can assume normality.

As the ANOVA test is significant, we can compute Tukey HSD (Tukey Honest
Significant Differences, R function: TukeyHSD()) for performing multiple
pairwise-comparison between the means of groups. When taking a look at
the output of the function we can conclude that the difference between
Amsterdam- Center with all the other districts gives a significant
p-value below 0.05. For all the other differences between districts,
there is no significant difference.

When looking at the confidence interval we did with 95\%, the outcome we
get is {[}0.00, 0.00{]}. This means that if you run your experiment
again you have a good chance of finding no difference between groups. It
being perfectly zero is very rare, but it is most likely due to the fact
of the lockdown being a rare circumstance and the corresponding time
frame.

\hypertarget{conclusion}{%
\paragraph{Conclusion}\label{conclusion}}

After analyzing the data we collected we can conclude that there is a
slightly change in price due to the second lockdown. However most of the
accommodations did not change prices (64.5\%). The second lockdown in
the Netherlands did affect the prices of the Air bnb accommodations in
Amsterdam. Not all accommodations did change their prices, only 35.5\%
did increase or decrease their price. The most common price changes
happened during and after the lockdown. The trend before the lockdown
was to keep the prices just the same, during lockdown landloards did
decrease their prices and after the lockdown they increased their prices
to a level that is higher than the price before lockdown. The location
of the accommodation did affect the prices. The districts that were
created for this research give a good overview. The `Center Amsterdam'
district changed the most during the lockdown. Accommodations further
from the `Center' did change less, the further from the Center, the less
the prices changed.

\hypertarget{repository-overview}{%
\subsection{Repository overview}\label{repository-overview}}

The repository consists of four folders (workflow, data, src, and gen),
and three files (.gitignore, README.md, and dprep-sick.Rproj). The aim
of the research project, instructions, running details and results are
communicated in README.md file where you are currently viewing. The data
being used in this project can be found in the data folder but can also
be accessed through the src folder,make file where the overflow of
running instructions for the analysis is also provided.

\hypertarget{running-instructions}{%
\subsection{Running instructions}\label{running-instructions}}

\hypertarget{required-software}{%
\subsubsection{Required Software}\label{required-software}}

In order to run the code without problems: - Install R and R studio - In
order to run the code a few additional packages within R are required
(these packages are also mentioned in the code when needed):

\begin{itemize}
\item
  install.packages(``googledrive''), install.packages(``readr''),
  install.packages(``dplyr''),
  install.packages(``tidyverse''),install.packages(``stringr''),
  install.packages(``ggpubr''), install.packages(``ggplot2''),
  install.packages(``car''), install.packages(``effectsize''),
  install.packages(``broom''), install.packages(``agricolae'').
\item
  Install Make. For the smooth reproduction of the workflow steps.
\end{itemize}

Explain to potential users how to run/replicate your workflow. Touch
upon, if necessary, the required input data, which (secret) credentials
are required (and how to obtain them), which software tools are needed
to run the workflow (including links to the installation instructions),
and how to run the workflow. Make use of subheaders where appropriate.

\hypertarget{running-the-workflow}{%
\subsubsection{Running The Workflow}\label{running-the-workflow}}

The Make file will run the worflow in the following order below. If
interested the steps mentioned below can be run seperately by the
makefiles connected in their own folders. (eg:../src/analysis
-\textgreater{} make.file).

\begin{itemize}
\tightlist
\item
  download\_data.R
\item
  data\_cleaning.R
\item
  analysis.R
\end{itemize}

\hypertarget{more-resources}{%
\subsection{More resources}\label{more-resources}}

Point interested users to any related literature and/or documentation.

\hypertarget{about}{%
\subsection{About}\label{about}}

This research is carried out in implementation of the course Data
Preparation and Workflow Management. This is a part of the Master
program of Marketing Analytics. The following authors contributed to
this research:

\begin{itemize}
\tightlist
\item
  \href{https://github.com/bramvdbemt}{Bram van den Bemt}
  \href{mailto:b.c.r.vdnbemt@tilburguniversity.edu}{\nolinkurl{b.c.r.vdnbemt@tilburguniversity.edu}}
\item
  \href{https://github.com/dogabayraktar}{Doğa Bayraktar}
  \href{mailto:d.b.bayraktar@tilburguniversity.edu}{\nolinkurl{d.b.bayraktar@tilburguniversity.edu}}
\item
  \href{https://github.com/Demidegroot}{Demi de Groot}
  \href{mailto:d.degroot@tilburguniversity.edu}{\nolinkurl{d.degroot@tilburguniversity.edu}}
\item
  \href{https://github.com/EllenB1}{Ellen van Berlo}
  \href{mailto:e.d.vanberlo@tilburguniversity.edu}{\nolinkurl{e.d.vanberlo@tilburguniversity.edu}}
\item
  \href{https://github.com/SamMes98}{Sam Messaoudi}
  \href{mailto:s.t.l.messaoudi@tilburguniversity.edu}{\nolinkurl{s.t.l.messaoudi@tilburguniversity.edu}}
\end{itemize}

\end{document}
